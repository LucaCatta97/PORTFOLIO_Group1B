\chapter{Risk}
\begin{figure}[h]
\centering
\includegraphics[width=0.8\textwidth]{Capitoli_Report/6.1_Risk.png}
\caption{\cite{picrisk}}
\label{fig:risk}
\end{figure}
\newpage
\section{Introduction}
The case study consists in the discussion of our position regarding the acceptability of the risks imposed by autonomous vehicles to vulnerable road users. This will be performed using the five ‘factors’ for the acceptability of risk presented in the lecture as a guiding structure.

The goal is \textit{to try to come to a clear conclusion} as to whether we think the risks to vulnerable road users are acceptable or not and, if that would not be possible, we will indicate why it is difficult or what would be necessary to be able to make the determination. We will start from the assumption that \textbf{while overall traffic safety significantly increases with the introduction of autonomous vehicles, safety for some vulnerable road users actually decreases slightly}.

\emph{(125 words)}

\section{Discussion}
\paragraph{The size of the risk}
To discuss about the size of this risk, two different views will be presented: a relative and an absolute one.

Talking about the relative size, of course in some car-oriented countries like the USA, the slight increase of risk to cyclists can be considered negligible looking at the great benefit that autonomous cars will give. However, in some other countries like the Netherlands, where cycling has an overall modal share of $27\%$ (\cite{enwiki:1057840626}), the weight of this change will be greater.
On the other hand, looking at this risk in absolute terms, considering the pretty straightforward fact that a cyclist is a person that wants, like a car user, to be safe during her/his trip, the size of this risk is something not negligible.

\emph{(126 words)}

\paragraph{The availability of alternatives with lower risk}
Considering that an increase in risk, however small, is not negligible, several alternatives are available which could help to reduce the risk for vulnerable road users, or at least to keep it at the same current level.

For example, we could think about the introduction of a road infrastructure that physically divides drivers from cyclists and pedestrians, reducing the number of interactions between them and consequently the number of accidents, thus increasing overall safety. But a solution of this type introduces non-negligible disadvantages, such as the time and money required for its implementation. Therefore, in evaluating the possible use of alternative solutions, it is necessary that all the different stakeholders are willing to sacrifice something to favor the safety of the other party and, ultimately, the introduction of autonomous driving itself.

\emph{(132 words)}

\paragraph{Whether/to what extent the benefits outweigh the risks}
Utilitarians would argue that, as car drivers represent the majority of traffic users, an increase in their safety at the expense of vulnerable road users would be justifiable. However, this is probably a too trivial approach to decide whose safety has to be sacrificed. Also, it is conventionally believed that vulnerable users are the ones that should be more protected on the road.

Considering these aspects, the benefits of autonomous vehicles would outweigh the risks only in the case where car users’ deaths and injuries would decrease so much, that a slight worsening in vulnerable users’ conditions would be acceptable.

\emph{(100 words)}
\newpage
\paragraph{The degree of informed consent (or equivalent) for the risks}
In theory, a technical development leading to an increased safety for car-drivers would easily find their consent while, on the contrary, this would not happen for vulnerable road-users. But car-drivers could also travel as vulnerable users and many pedestrian could drive their cars, so not necessarily all car-drivers would agree for a solution like that and vice versa.

Some people, like elders and disabled, cannot avoid being vulnerable, so their consent should be kept in consideration more attentively. But, since this measure would affect positively the whole society, only a very wide aware opposition of such stakeholders should influence the putting in place of this measure.

(110 parole)

\paragraph{The fair distribution of the benefits and risks}
The distribution of benefits and risks can rarely be fair. This is due to the impossibility to reach always the optimal solution; we will need to find a compromise, otherwise we will be stopped. This process will favor the categories with greater influence in the decision-making process.

In this case, the vulnerable road user will necessarily reach a compromise to avoid a riskier approach. At the same time, the users of autonomous vehicles will reach a compromise accepting some risks to avoid a stop of the usage of autonomous vehicle. The distribution of benefits and risks both depends on how much influence the different parties have in society, making the process unfair.

\emph{(113 words)}

\section{Conclusion}
Summarizing what has been said, the imposition of a higher risk for already vulnerable users is hardly acceptable. By the way, before coming to an exhaustive conclusion, some other considerations must be done. It must be kept into account that a more performing traffic flow would lead to significant benefits for the whole society: more effective services, less rescue time for emergency vehicles, less pollutant emission and other improvements.

We can understand that there would be many aspects that would counterbalance the increased risk when considering the vulnerable users welfare. So, to make a choice, we would need data about lives of people that would be saved thanks to those "externalities" and also how to consider other well being improvements indicator, such as economic savings, higher quality of life and similar.

\emph{(131 words)}

\emph{(700 total words)}

\begin{comment}
\begin{thebibliography}{99}

\bibitem [1]{p1}

Wikipedia contributors.

\textit{Cycling in the Netherlands.}

\href{https://bit.ly/3pvCNY7}{\textbf{Wikipedia, The Free Encyclopedia}} Accessed on 4th December 2021.

\bibitem [Hayenhjelm, 2012]{p2}

Hayenhjelm, M., \& Wolff, J. (2012).

\textit{ The Moral Problem of Risk Impositions: A Survey of the Literature.}

\textbf{European Journal of Philosophy}, 20(S1), e26–e51.

\end{thebibliography}
\end{comment}