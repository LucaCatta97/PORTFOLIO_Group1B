\chapter{Justice}
\begin{figure}[h]
\centering
\includegraphics[width=0.8\textwidth]{Capitoli_Report/3.1_Justice.png}
\caption{\cite{picjustice}}
\label{fig:justice}
\end{figure}
\newpage
\section{Introduction}
The case study regards the comparison of two possible theories to apply in order to decide whether to build a new highway bridge or the BART tunnel in the San Francisco Bay Area. The approaches we selected are \textit{utilitarianism} and \textit{libertarianism}.

\emph{(41 words)}

\section{What are the most important implications of using the selected approaches in the choice between the BART tunnel and the highway bridge?}

{\em{\bfseries Libertarians}} think that people are able to pursue their best interests when left free to act. Consequently, any intervention of the authority should be limited; moreover, in a libertarian society, infrastructures and services would be run completely by private actors pursuing only their personal income, so the most suitable project would be the highway bridge. In fact, it is much cheaper than the BART, resulting in a lower CAPEX for the (private) construction company and it is also the alternative that would have a greater market share. This solution allows to apply fares for the transit, so with the assumption that the two infrastructures would generate the same income, a less expensive solution would create more profit, being the alternative that a company in a libertarian system would naturally choose. Obviously this would not take into account the poorest people, who are unable to afford the expenses, leading to a non-optimal distribution of opportunities and generating an improved quality of life only for a minority of the population. We can understand that for libertarians, only taxpayers and contributing people’s opinion should be taken into account. 

{\em{\bfseries Utilitarians}} instead aim to maximize the total benefit quantified using economic terms, considering as equally important everyone’s opinion on the issue. The counter effect of this strategy is the concentration only on necessities shared by the majority of people or even worse, only on few people able to obtain great improvements, counterbalancing eventual relative losses of the rest of the population. According to this approach, the most suitable project would still be the car bridge, since the majority of people (72\%) seems to prefer the private transport. Again, the most disadvantaged citizens would anyway be penalized, needing to afford a car to get access to the service.

{\em{\bfseries In general terms}}, consequences of a libertarian approach risk to be harsher with respect to the utilitarian ones, because they are inherently unfair, distributing opportunities unequally among the population and preventing disadvantaged people from flourishing. Utilitarianism that is still a non-optimal solution, takes someway into account everyone’s need and tries to pursue a generalized development, even if often at minorities' expenses. Another good point in favor of utilitarianism, is the evaluation and quantification of bad consequences of policies assuring a higher benefit and so, theoretically, the possibility to refund the underprivileged, that could become convenient if misery begins to widespread, threatening even majorities' well being.

\emph{(399 words)}

\section{What are the most important differences between these two theories of justice applied to this case in terms of:}
\subsection{Practicality}
The utilitarian approach addresses how to distribute the greatest good among all the members and it is the basis of the {\rmfamily Cost Benefit Analysis} (\texttt{CBA}), that analyzes costs and benefits for each alternative to prioritize the most convenient ones, allowing a quantifiable control of the different objectives. Practicality can be quantified with different indicators that need to be maximized in order to produce the greatest utility. 

Libertarianism instead does not allow any external intervention, emphasizing the individual freedom and the financial feasibility. This leads towards the solution that is the most economically convenient, but could not be the one that is shared by the majority of the users, nor the socially optimal one.

\subsection{Fairness}
From this point of view, the most useful approach is utilitarianism because, despite it can address only the activities used by majoritarian classes and eventual bad externalities will fall only on minoritarian classes, it is also true that these problems can be solved by addressing in \texttt{CBA} those externalities. Instead, libertarianism will solve no problem in case of market failures. 

An example, remaining in the case of infrastructure investment, can be the case of internet home connection in Italy. Due to a free market libertarian approach, we have towns served only by \texttt{ADSL} from a single infrastructure operator, because they have less users of the internet service, while other cities are served with \texttt{FTTH} with more than one infrastructure operator, being easier for them to make a profit.

\emph{(241 words)}

\emph{(681 total words)}

\begin{comment}
\newpage
\begin{thebibliography}{99}

\bibitem[Justice, 2017]{p2}
Pereira et al. (2017)

\textit{Distributive justice and equity in transportation.}

\textbf{Transport Reviews, 37(2)}

\bibitem[Investment Decisions, 2020]{p3}
Kevin DeGood (2020)

\textit{Infrastructure Investment Decisions Are Political, Not Technical.}

\textbf{Center for American Progress}

\end{thebibliography}
\textit{Document write with \LaTeX. Template founded on Overleaf} (\textbf{Copyright (c) 2020 George Kour}).
\end{comment}