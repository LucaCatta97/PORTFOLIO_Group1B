\chapter{Privacy}
\begin{figure}[h]
\centering
\includegraphics[width=0.8\textwidth]{Capitoli_Report/5.1_Privacy.png}
\caption{\cite{picprivacy}}
\label{fig:privacy}
\end{figure}
\newpage
\section{Introduction}
The case study regards the connection between the philosophical analysis from the last workshop to ethical design, practicing the translation from the conceptual to the practical and viceversa. The analysis is based on the norms of appropriateness and of distribution defined by \textbf{Zimmer} \cite{Zimmer2005SurveillanceTechnologies}. 
\begin{table}[h]
\centering
\begin{tabular}{|c|l|l|}
\hline
\textbf{Value}                                             & \multicolumn{1}{|c|}{\textbf{Norms}}                                                                                                                            & \multicolumn{1}{|c|}{\textbf{Design requirements}}                                                                                              \\ \hline
\multicolumn{1}{|c|}{\multirow{12}{*}{\textbf{Privacy}}}   & \multicolumn{1}{l|}{\multirow{2}{*}{A1: acquired data contain only the image}}                                                                                 & \multicolumn{1}{l|}{\multirow{2}{*}{No metadata for images taken from passengers}}                                                               \\
\multicolumn{1}{|c|}{}                                     & \multicolumn{1}{l|}{}                                                                                                                                           & \multicolumn{1}{l|}{}                                                                                                                           \\ \cline{2-3} 
\multicolumn{1}{|c|}{}                                     & \multicolumn{1}{l|}{\multirow{2}{*}{\begin{tabular}[c]{@{}l@{}}A2: don't merge images \\\qquad with data from other sources\end{tabular}}}                     & \multicolumn{1}{l|}{\begin{tabular}[c]{@{}l@{}}Impossibility of making query \\ in other dataset of the instrument\end{tabular}}               \\ \cline{3-3} 
\multicolumn{1}{|c|}{}                                     & \multicolumn{1}{l|}{}                                                                                                                                           & \multicolumn{1}{l|}{\begin{tabular}[c]{@{}l@{}}Database with only photo \\ of the inhabitants\end{tabular}}                  \\ \cline{2-3} 
\multicolumn{1}{|c|}{}                                     & \multicolumn{1}{l|}{\multirow{2}{*}{\begin{tabular}[c]{@{}l@{}}A3: renew resident images \\ \qquad in the database at regular time\end{tabular}}}                & \multicolumn{1}{l|}{Automatic reset of image dataset}                                                                                           \\ \cline{3-3} 
\multicolumn{1}{|c|}{}                                     & \multicolumn{1}{l|}{}                                                                                                                                           & \multicolumn{1}{l|}{\begin{tabular}[c]{@{}l@{}}Acquire the photo provided \\ for the Identity card\end{tabular}}                         \\ \cline{2-3} 
\multicolumn{1}{|c|}{}                                     & \multicolumn{1}{l|}{\multirow{2}{*}{D1: data not shared}}                                                                                                       & \multicolumn{1}{l|}{\multirow{2}{*}{Local dataset}}                                                                                             \\
\multicolumn{1}{|c|}{}                                     & \multicolumn{1}{l|}{}                                                                                                                                           & \multicolumn{1}{l|}{}                                                                                                                           \\ \cline{2-3} 
\multicolumn{1}{|c|}{}                                     & \multicolumn{1}{l|}{\multirow{2}{*}{D2: data kept only for limited amount of time}}                                                                             & \multicolumn{1}{l|}{\multirow{2}{*}{\begin{tabular}[c]{@{}l@{}}Automatic reset of the memory\\  used to store acquired data\end{tabular}}}     \\
\multicolumn{1}{|c|}{}                                     & \multicolumn{1}{l|}{}                                                                                                                                           & \multicolumn{1}{l|}{}                                                                                                                           \\ \cline{2-3} 
\multicolumn{1}{|c|}{}                                     & \multicolumn{1}{l|}{\multirow{2}{*}{\begin{tabular}[c]{@{}l@{}}D3: if data required by authority \\ \qquad only with mandate (according to law)\end{tabular}}} & \multicolumn{1}{l|}{\multirow{2}{*}{\begin{tabular}[c]{@{}l@{}}End to End cryptography \\ between the camera and the storage/analysis system\end{tabular}}} \\
\multicolumn{1}{|c|}{}                                     & \multicolumn{1}{l|}{}                                                                                                                                           & \multicolumn{1}{l|}{}                                                                                                                           \\ \hline
\end{tabular}
\label{tab:my-table}
\end{table}
\emph{(48 words)}

\section{Conceptualization of privacy}
\textbf{Privacy} is the condition in which all the entities not involved in yourself interest, or who did not receive your own acceptance, \textit{are deprived from accessing your personal data, such as habits and living area}. To do so, data collection from systems must always respect \textbf{contextual integrity} and its norms:
\begin{itemize}
    \item In \textbf{norms of appropriateness}, this concept declines in the given context considering the collection only of the amount of data that are needed to perform facial matching, and not any other information.

    \item \textbf{Norms of distribution} may instead concern the fact that the data collected must not be shared with other entities outside of the bridge automatic toll system, to prevent eventual loss of users’ privacy.
\end{itemize}

\emph{(115 words)}

\section{Does the solution present privacy concerns? Discuss for each of Reiman’s risks of privacy loss.}

\paragraph{Extrinsic Loss of Freedom}
It is defined as \textit{"all those ways in which lack of privacy makes people vulnerable to having their behavior controlled by others"}.

In this case, the risk is minimized by the fact that data are saved locally. But, we still could have some conditioning on people's travel habits: in fact, they could feel observed only because of the presence of cameras, like drivers that slow down approaching a speed trap, even though they know it is turned off. By the way, this risk could be considered acceptable, especially if we consider that the population will be properly informed about the functioning of the system and of the data collected.
\newpage
\paragraph{Intrinsic Loss of Freedom}
According to Reiman, it is the \textit{"ways in which denial of privacy limits people's freedom directly"}.

According to that, we do not face out any problems with the only exception of the presence of the toll system. Indeed, it limits the privacy of the single citizen, being forced to share his/her image to be recognizable and to have free access to the island. However, this intrinsic limitation of freedom is negligible especially for residents, since they are not required to pay the fee.

\paragraph{Symbolic Risk}
In the conception of the panopticon, it is defined as \textit{"a kind of draining off our individual sovereignty away and outside of us into a single center"}.

In the present case, this risk does not seem to be relevant, in relation to the previously mentioned requirements and to the control, that is limited only to that specific geographical location. Nevertheless, diffused control on travel data could be a problem, because it could preclude individuals from their authority to withdraw from scrutiny of others.

\paragraph{Psycho-Political Metamorphosis}
Reiman, talking about the development of the panopticon states that, when continuously observed, \textit{individuals tend to lose their peculiarities and behave the same, with a loss in the different shades of personalities}.

We think that this consideration, however, does not apply to the implementation of this solution: the control is performed only for the sake of making people respect a rule and the data is kept locally for a short time. So, this technology does not collect enough information to condition people's psychology.

\emph{(359 words)}

\emph{(522 total words)}

\begin{comment}
\begin{thebibliography}{99}

\bibitem[Privacy, 2017]{p1}
Reiman, J. H. (2017)

\textit{Driving to the panopticon: a philosophical exploration of the risks to Privacy Posed by the Highway Technology of the future.}

\textbf{Privacy, 11(1)}

\bibitem[Ethics and Information Technology, 2005]{p2}
Zimmer, M. (2005)

\textit{Surveillance, privacy and the ethics of vehicle safety communication technologies.}

\textbf{Ethics and Information Technology, 7(4)}

\end{thebibliography}

\textit{Document write with \LaTeX. Template founded on Overleaf} (\textbf{Copyright (c) 2020 George Kour}).
\end{comment}