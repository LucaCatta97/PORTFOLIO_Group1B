\chapter{Why should an engineer rely on ethics as well as numbers?}

\begin{flushright}
Individual Reflection by Jacopo Elia Pometto (10521596)
\end{flushright}

\paragraph{Introduction}
Modern technologies and societies seem to be increasingly focused on one thing: \textit{economical profit}. This happens without taking in consideration \textit{the possible consequences that third parties may suffer} due to the actions taken by companies in order to pursue their goal.
For this reason, a mobility engineer, who should help and accompany companies in the next ecological transition, should be aware of the benefits of a more \textit{ethical approach} in this discipline.
In this sense, the \textit{Ethics for Transportation} course offers a general look at the methods to be used for this type of reasoning, adding them to the “cold” numerical analysis, which is still fundamental.

\paragraph{Useful method for an ethical approach}
Throughout the course, various aspects of ethics related to the world of transport were addressed.

As a future engineer, the ones I found most interesting were \textbf{Value Sensitive Design}, \textbf{Justice in Transportation} and \textbf{Responsibility}. In particular, the latter two will be fundamental when decisions have to be made: it must be ensured that progress is equitable and optimally distributed among all people.

From my point of view, looking at the \textit{five theories} proposed by \cite{Pereira2017DistributiveTransportation}, \textbf{utilitarianism} is the one that would probably guarantee the best distribution of justice, reminding us, however, that the goal is to maximize not only \textit{profit} for the companies, but also \textit{welfare} for people. In this sense, a \textbf{Multi Criteria Analysis} is fundamental, as it allows to take into consideration different aspects, each weighed according to its relative importance, and to derive the project that best satisfies them all.

As I said above, another important aspect for me is \textbf{responsibility}: in fact, for a system to be fair it is essential that responsibility is well distributed among the various players involved, whether they are decision makers, stakeholders or regulators, and above all that it is well defined, also to prevent the emergence of problems such as the one identified by \cite{VANDEPOEL2018MoralHands.}. 

In this case, I still don't have a clear idea of what could be the best approach between the different types of responsibilities identified by the \cite{EuropeanCommission2020EthicsEU}: \textbf{culpability} and \textbf{accountability} are certainly fundamental, to clearly delineate the boundaries of responsibilities, from an ethical, legal and social point of view. However, one could also consider the idea of a system based on responsibility as a \textbf{virtue}, although the risk of finding oneself within a \textit{responsibility gap} is higher. In the end, the best solution would probably be a sort of overlap of the three previously mentioned.

All considered, the \textbf{VSD} (\cite{Watkins2021UsingSystem}) fits perfectly in my opinion. Its goal is in fact \textit{to make moral values an integral part of technological design, guaranteeing the mixture of engineering and ethical aspects}. Furthermore, as we understood during the lessons, engineers have also \textit{the responsibility to try to think in advance about the impact of their activity, while promoting right values through their design choices}.
\newpage
\paragraph{Conclusion}
To conclude, this course was fundamental for me to better understand how necessary it is to provide an ethical background to all those figures who will play an active role in the decision-making processes, training engineers capable of reflecting on issues beyond the numerical and technological aspect. 

It is right for companies to pursue profit, otherwise they would have no way to support themselves, but I find that they would also benefit if they took more into consideration the ethical effects on the whole society.

To close this brief reflection, I would like to quote the words used by \cite{eticatrasporti}, in reference to the ecological transformation that the world of mobility is preparing to face, since they perfectly summarize my thoughts:
\begin{displayquote}
\textit{"If we have to do an E-inversion, let it be an Ethical inversion as well."}
\end{displayquote}

\emph{(612 words)}