\chapter{Relevance and ethical implications of risk analysis in modern societies’ professions}

\begin{flushright}
Individual Reflection by Alessandro Barbero (10536528)
\end{flushright}

\paragraph{Risk analysis for modern decision makers}
Nowadays technological development has disclosed many opportunities for advanced societies that some decades ago wouldn't have been imaginable but at the same time a serious counter effect has emerged. The deep permeation of complex devices into people’s lives has made them vulnerable to a whole new set of risks; for this reason modern professions, especially engineers and researchers, must deal with this issue that, as we will see, is dense of ethical implications. Moreover they must be aware also of the limits of the ethical theories that are being applied and also of their personal evaluations, and the contents presented in this ethics course are significantly helpful for their reasoning. 

Up to now, the most widespread tool used to assess if a risk imposition on the population is acceptable has been the CBA, from the consequentialist approach (\cite{Pereira2017DistributiveTransportation}). It certainly represents a useful tool also for future engineers but they will need also to be aware of its limits (such as the lack of attention to eventual uneven spread of benefits on people etc...) in order integrate it with other more advanced approaches.

It must be remembered that researchers and engineers, just like anyone else, may have ideas and interests making them preferring certain policies upon others. In this cases it is necessary from them the highest impartiality and the application of the Rawls' contractualism (\cite{Pereira2017DistributiveTransportation}) fits perfectly for this purpose. Attention to people's rights and wills must be considered but not blindly; in fact if people could impose a veto on every action affecting their lives in a way the consider as negative, we would fall into the paralysis problem (\cite{Hayenhjelm2012TheLiterature}).

\paragraph{Relevant ethical implications of risk}
There are many implications deriving from risk; in order to give a glance of the complexity of the issue, some have been presented.

One thing can be easily agreed: risk is always to be reduced when no implications arise.  Unfortunately usually this condition doesn’t happen and a risk professional must predict (as much as possible) eventual side-effects and evaluate them before establishing a new policy. For example in order to reduce risks often economical costs must be paid and it must be evaluated who should sustain them (only involved stakeholders from a liberal point of view, or every citizen according to her wealth etc..), other times it could be required to evaluate if freedom of people itself should be acceptable to reduce or not.

One more aspect to analyze is the responsibility that risky decisions and their eventual bad outcomes imply. Obviously most of the times, policies, actions and their consequences are the result of a group of people's cooperation, this leads to the need for engineers to be aware of the collective responsibility (\cite{VANDEPOEL2018MoralHands.}) which they have together with the rest of their company. An inherent duty of their profession must be to try as much as possible to identify eventual negative scenarios and to prevent them as implied in the concept of forward responsibility (\cite{VANDEPOEL2018MoralHands.}). How responsibility must be distributed among actors facing the problem of many hands (\cite{VANDEPOEL2018MoralHands.}), is a another issue that need to be cautiously tackled.

\newpage
\paragraph{Conclusions}
As shown from an ethical perspective, the risk analysis is crucial and presents many implications that widespread on several other fields and each on them impacts the decision making process of the policy maker. Only a deep awareness of these topics allows to face them in the proper manner helping to provide benefits to the whole society which is the real main purpose of the engineer profession itself.

\emph{(599 words)}