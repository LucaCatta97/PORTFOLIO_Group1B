\chapter{Professional ethics as a guideline for a project development}

\begin{flushright}
Individual Reflection by Giovanni Valtorta (10528573)
\end{flushright}

\paragraph{Introduction}\
Ethics, and more strictly speaking \textbf{professional ethics} is something that is vital to consider in every professional role. The higher the importance of the role, the more important professional ethics becomes.
To discuss how the topics discussed in this course would be an added value to a career in engineering, I would like to outline the implications that they could have in different phases of an example project. 

\paragraph{Design phase}\
Since the first phases of the project, \textbf{Value Sensitive Design} \cite{Watkins2021UsingSystem} would be something of vital importance in a world that is getting increasingly complex. 

This \textbf{complexity} is due to the presence of multiple actors and stakeholders, each with different values to be pursued through the implementation of the project. 

\textbf{VSD} can be the answer to this, to put it into engineering terms, multi-variable problem. The variables being the different values put onto the table during the design phase, VSD can try to provide the best solution that tries to encompass every different aspect. 

But the question rises spontaneously: \textit{\textbf{is the output of VSD something that is optimal for all the actors? }}

Of course not, it would be impossible to completely avoid Value tensions between actors. 

So in this phase,  \textbf{Justice} becomes the focus of the analysis: \textit{why have some values been left out of the analysis? }

Between the multitude of theories that have been presented, I think that \textbf{utilitarianism} \cite{Pereira2017DistributiveTransportation}is the most applicable one in the engineering world to answer the posed question. 

Utilitarian principles take the form of \textbf{Cost Benefit Analysis}, which is actually the guiding principle of most projects. CBA can  be further extended in \textbf{Multi Criteria Analysis}, in which not a single Key performance index is considered but a matrix of them is created, different weights are assigned to each KPI and then the utility function is computed on the basis of those values. 

A utilitarian approach, so, can provide an answer to the question posed before. 

Some values have been left out of the analysis because simply those were not in the path of pursuing the greatest interest for most of the whole community, which is something that, for infrastructural project, is vital. 

At the end of the design phase, the outcome is the solution which represents the best \textbf{compromise}.

\newpage
\paragraph{Execution phase}\
A compromise, however, is something that would certainly make some actors unhappy in the executive phase of a project, so the last theme that needs to be stressed is \textbf{Responsibility}.  

Every decision is taken by an entity, being that the company, a single department or even a single person. In this environment with many actors, the so called Problem of Many Hands \cite{VANDEPOEL2018MoralHands.} emerges. 

What is needed to avoid the ping pong of responsibility if something happens, is a shift in how responsibility itself is perceived.  

What would be interesting, instead, would be to shift from a view of responsibility as \textbf{Culpability} \cite{EuropeanCommission2020EthicsEU} to a view of responsibility as a \textbf{Virtue}, to create an environment in which everyone is eager to do his/her best and not be afraid to take responsibility for that action.

\paragraph{Conclusions}\
To conclude, I want to say that I was very curious about this course since too rarely in the Italian school system scholars are exposed to philosophical problems.

The  view of engineering being pure determinism, where every cause has an effect and each phenomenon is described by laws or formulas, can surely be seen as a limit to the free thought experiment; so it  was fun, for once, to let the mind roam free outside of the schemes.

\emph{(596 words)}