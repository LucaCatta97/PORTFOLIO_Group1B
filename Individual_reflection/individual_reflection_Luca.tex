\chapter{Ethics: a matter always rejected but that will be useful for engineer}

\begin{flushright}
Individual Reflection by Luca Cattaneo (10521219)
\end{flushright} 

%%%% SCHEMA SCRITTURA
%
% Inizio: percorso dei workshop difficile 
%
% Poi parlare del cardine che è il VSD rispetto futuro professionale
%
% pensare a qualcosa su Justice e/o altro boh
%
%
%-----------

\paragraph{Introduction}
I want to start this reflection by saying that this path is one of the most difficult that I made during my permanence at the Politecnico di Milano. I think this is due to the feature of this course: It is completely different from a typical course of engineering. I see that this perception is very common and I agree with \cite{eticatrasporti} when he tells that this rejection to ethics and philosophy is due to the way they are been taught in Italy (e.g.: we learned the authors but no one taught us how to apply their thinking in high school) and also to the separation between scientific and humanities disciplines started from the nineteenth century.

But as we will see in the next paragraphs the topics that we saw during the course are very important.  

\paragraph{The importance of VSD}
As it is already been said in the first report (\ref{vsd}) the use of a framework such as the VSD is essential at least to assess all the possible problems of an infrastructure; as engineers we will have to make decisions on a very impacting systems.

An example that support this is the so called Turin–Lyon high-speed railway that will allow to have a railway connection of 4 hours from Milan to Paris (the actual and "new" connection from Milan to Paris recently activated takes 6 hours). This can be seen as a huge progress from a scientific point of view but to implement such infrastructure we need basically to pierce mountains and divide in two part a valley.

Another argument is provided by \cite{eticatrasporti} when he wonders the reason why engineers should be concerned with philosophy and applied ethics and for him, this is due to the fact that engineering is a political profession. He justifies this position by saying that designing technological systems means designing the functioning of the world and contributing to forming people's behaviors, political relationships and the relationship with the environment.

I agree with \cite{eticatrasporti} but someone can argue that technologies are neutral as we see in the \emph{instrumentalist} approach.
\cite{eticatrasporti} said that the technological projects are never neutral, they always reflect the ideas, principles and values of those who design them.  So he continues and affirm that it is therefore essential that engineers are trained to reflect on the ideas, principles and values that more or less explicitly incorporate or would like to incorporate in their projects.

And this is the reason why we need framework such as VSD.

\paragraph{The challenge of Ethics for Transportation and in general for Engineering}
Another interesting thing, as said in \cite{Pereira2017DistributiveTransportation}, is that the "historical" theories of justice are not so easy to adapt in transportation contest.
This, I think, is more a challenge rather than an obstacle. Why? Because both engineering and ethics can improve each other if they will interact (as said another time by \cite{eticatrasporti}).

\paragraph{Conclusion}
I started this reflection telling that Ethics is not so easy to apply and to study for an engineer but after this little reflection I can affirm that Ethics will be a fundamental part of (an) engineer's professional life. 

\emph{(502 words)}