\section{Summary}
In general, \textit{justice} can be related to the identification and distribution of benefits and duties, and to how fair the processes to enable this distribution are. The paper provides an overview of five justice theories that can be linked to transport and accessibility subjects, and these theories are:
\textit{\begin{itemize}
        \item utilitarianism;
        \item libertarianism;
        \item intuitionism;
        \item Rawls’ egalitarianism;
        \item capabilities approach.
        \end{itemize}}

All those perspectives face justice with different focuses and limits, which translate in different accessibility policies. The author conclude that the most sustainable strategies are Rawls' egalitarianism and capability approaches, since combined they manage to overcome criticalities emerging from other strategies.

\emph{(96 words)}

\section{Interesting concept}
The concept we found most interesting is the insight on political philosophy for transport policy and accessibility.
This application is the most useful guide for mobility engineers, because it can guide them in the management of projects for new or already existing infrastructures.

However, some of those theories can't always be applied in reality. For example, applying intuitionism in a context where the community cannot understand the infeasibility of some policies, can mean that some other alternatives aren't even considered, even if they could offer a better solution to the problem.

Another interesting point is that it can be difficult for an engineer to apply in its work a justice theory that differs from the one he personally believes. One example can be applying utilitarianism in life, chasing personal fulfilment regardless of others' opinions and needs, but being obliged to consider a more egalitarian approach while working, for instance, on an infrastructure project.

\emph{(153 words)}

\section{Trouble understanding}
\subsection{Critique against utilitarianism}
It focuses on the assumption that maximising the benefit would probably cut out the most fragile and weak components of society, since it ignores the secondary effects of an improvement of the majority wealth also on other social group, for example, more investments and consequently more job opportunity.

This view is limited, because it considers \textit{Cost Benefit Analysis} (CBA) as the only way to apply utilitarianism and hence lacks in considering other techniques like \textit{Multi Criteria Analysis} (MCA) which, despite being more complex than simple CBA, offers a wider spectrum of considered criteria, weighting them according to their importance.
Utilitarianism with MCA can move extremely close to egalitarianism, in the sense that it considers and weights according to their priority the different points of view of different stakeholders.

\emph{(128 words)}

\emph{(377 total words)}

\newpage
\begin{thebibliography}{99}

\bibitem[Justice, 2017]{p1}
Pereira, R. H. M., Schwanen, T., \& Banister, D. (2017)
\textit{Distributive justice and equity in transportation.}

\textbf{Transport Reviews, 37(2)}
\end{thebibliography}

\textit{Document write with \LaTeX. Template founded on Overleaf} (\textbf{Copyright (c) 2020 George Kour}).