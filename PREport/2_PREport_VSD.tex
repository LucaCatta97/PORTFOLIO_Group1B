\section{Summary}
The transportation system in modern society is very important but it is also linked to many counter effects such as pollution and safety issue. Consequently, innovative modes of transport must be developed but it could be difficult to make the society accept such changing, because people’s attitudes deeply influence the mode choice and value tension could emerge by different stakeholders.

Since a sustainable development is possible only if all stakeholders and their respective values are respected, the implementation of a systematic approach, like VSD, is required to account for the public’s true needs and desires.

\emph{(95 words)}

\section{Interesting concept}
What is particularity interesting in the article is the concept of value tension.
With value tension it is meant a contrast between different stakeholders’ values, between individual and societal priorities and between different individual convictions. We consider that this aspect is fundamental in the identification of the core values according to which all technologies must be developed, because it allows us to be aware that we will never be able to tackle all the possible necessities present in the society with only one solution, so sometimes we must renounce to foster a certain value to develop properly some others. 

There are many different situations in which this concept is applied. For example, a strong tension can be found between people willing to impose to the whole population the vaccination against the COVID-19 disease (with the aim of protecting all the fragile categories of that population) and the opposite party manifesting against a violation of their natural right to freedom of choice.

\emph{(162 words)}

\section{Trouble understanding}
\subsection{Critique to monetary value}
What we found difficult to understand is the critique, expressed by the article, of the monetary value as the only aspect that matters for the community.

In fact, even if the pursue of money alone can’t be considered enough for a satisfying development of the society, at the same time monetary terms could be a good solution to quantitatively express the importance of an aspect upon the others, linking to each of them a certain value depending on different parameters, such as the number of stakeholders involved and eventual counter-effects of implementation.

What is lacking in the article is a convincing presentation of an alternative to this issue which is, in other terms, the resolution of eventual value tension in the decision process.

\subsection{What value can be neglected?}
Another topic difficult to understand is, in the analysis, what values could be neglected and what other should prevail. In the same way, it’s unclear if it is better to fully develop only a certain value leaving all the alternatives or if it preferable to find a common ground trying to satisfy all the parties but not completely.

\emph{(182 words)}

\emph{(439 total words)}

\newpage
\begin{thebibliography}{99}

\bibitem[Watkins, 2021]{p1}
Watkins, K. E. , (2021)
\textit{Using value sensitive design to understand transportation choices and envision a future transportation system.}

\textbf{Ethics and Information Technology, 23(1)}

\end{thebibliography}

\textit{Document write with \LaTeX. Template founded on Overleaf} (\textbf{Copyright (c) 2020 George Kour}).