\section{Summary}
In \cite{p2} de Poel gives us an analysis about the \textbf{problem of many hands}, deepening also the definition due to ambiguous therm such as morally responsible, the collective and collective moral responsibility.

Also, he defines three types of collectives (\textit{organised, joint action and occasional collection}) and the different conditions for the different responsibilities.

In the end, he addresses the \textbf{reducibility thesis} and gives a stronger version of it, telling us that both RT and PMH can be accepted.

\emph{(83 words)}

\section{Interesting concept}
The \textbf{PMH} is a particular situation where authorship of a certain situation or outcome can't be attributed to a specific individual, but to a whole community in which none of the members is directly responsible of it, since it is the collection of their individual intervention.

A first example can be found in the Italian public transportation service not providing an acceptable level of service. In fact, inefficiencies of the service cannot be attributed to a single individual, but only in general to a collective group, like commuters not paying the ticket, or the public party inserting unskilled workers and making favoritism.

Another example, in positive terms, is the collective worth of members of a football team winning the Coppa Del Nonno. In fact all players are necessary to achieve such a goal, but none of them is enough. Moreover, also the coach contribution is fundamental, together with fitness trainers and to the point that even fans cheering at the stadium and paying the ticket, give some kind of contribution to the final outcome. It is anyway clear that more relevance is given to the people, on the basis of their involvement and non replaceability in the process.

\emph{(199 words)}

\section{Trouble understanding}
\subsection{Responsibility-as-accountability} This is a misunderstanding about the term: actually, if we translate \textit{responsibility} and \textit{accountability}, they result to be synonyms in the Italian language, and so we do not understand the differences that the author wants to provide.

\subsection{Strong Reducibility Thesis}
Another doubt is about \textbf{SRT}: how can all members be considered equally responsible for an event when the degree of awareness and the intention may be different among them?

\emph{(66 words)}

\emph{(348 total words)}

\newpage
\begin{thebibliography}{99}

\bibitem [Horizon, 2020]{p1}

Horizon 2020 Commission Expert Group to advise on specific ethical issues raised by driverless mobility (E03659).

\textit{Ethics of Connected and Automated Vehicles: recommendations on road safety, privacy, fairness, explainability and responsibility}.

\textbf{Publication Office of the European Union: Luxembourg}.

\bibitem [The problem of many hands, 2015]{p2}

I. van de Poel, L. Royakkers, S. D. Zwart (Eds.) (2015)

\textit{The problem of many hands}.

\textbf{Moral Responsibility and the Problem of Many Hands (pp. 50–92). Routledge}.

\end{thebibliography}

\textit{Document write with \LaTeX. Template founded on Overleaf} (\textbf{Copyright (c) 2020 George Kour}).