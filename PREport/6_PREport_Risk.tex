\section{Summary}
In \cite{p1} Hayenhjelm \& Wolff expose the different ethical approaches to risk. After an historical background and a contextualization in our society, they try to understand what should we mean by \textbf{risk}, providing different definitions.

They also identify different strategies to approach the ethics of risk: consequentialism, deontological approach, consent, contractualism, prima facie right, the precautionary principle, conventionalism and proceduralism.

By the way, all these theories present critical aspects that must be taken into account, since for society safety and outcomes of technological development are important but often incompatible and compromises must be found, trying to guarantee an acceptable level for each of them.

\emph{(107 words)}

\section{Interesting concept}
\textbf{The problem of paralysis} continually recurs in different aspects of life and can be summarized as \textit{"do not accept risk at the cost of stopping everything"}. If we assume that every citizen should have the right to impose a veto to every activity of others that constitutes a risk for his health or interest, we should deny almost all human action, since it is almost impossible to avoid to influence in any way others when performing also the most fundamental activities for our well being and sustenance. 

It can be seen in Italy's management of COVID-19 pandemic, especially during the months of March and April 2021: in fact, the most important thing was to reduce the risk of contracting the virus, even at the cost of sacrificing the country's economy and social relationships between individuals.

A contrary example is the imposition of speed limit at 130 km/h on the Italian highways: even if a reduction of top speed would certainly lower the number of accidents and deaths, at the same time it would affect negatively traffic management, with all the disadvantages coming from it.

Another interesting aspect is the \textbf{consent} concept.
In a few words, the author(s) states that if it is possible to prove that everyone facing a risk of harm is consenting to that risk, the moral problem of risk imposition no longer stands. However, this view lets some problems emerge in situations in which the consent to something is linked with a probability of risk.

To provide some examples, linking to women's rights, wearing something that can be considered "provocative" by some obsolete and man-oriented standards, does not imply that one woman is willing to be verbally, or even worse, physically molested.

\emph{(289 words)}

\section{Trouble understanding}
\subsection{Fair distribution of risks}
The paper assesses this problem, but we have not understood what the authors would mean with "fair": there is not a way in which we can say that the distribution of risk is objectively fair, because each individual, from his/her point of view, will always have a subjective definition of "fairness". So, how can we precisely define this fair distribution?

\subsection{Difference between tolerable and acceptable risk}
In terms of risk management, the concepts of tolerable risk and acceptable risk are mentioned. However, it is not stated a clear difference between the two, and why only the acceptable risks can be used as a benchmark for new risks.

\emph{(102 words)}

\emph{(498 total words)}

\newpage
\begin{thebibliography}{99}

\bibitem [European Journal of Philosophy, 2012]{p1}

Hayenhjelm, M., \& Wolff, J. (2012).

\textit{The Moral Problem of Risk Impositions: A Survey of the Literature.}

\textbf{European Journal of Philosophy}, 20(S1), e26–e51.

\end{thebibliography}

\textit{Document write with \LaTeX. Template founded on Overleaf} (\textbf{Copyright (c) 2020 George Kour}).