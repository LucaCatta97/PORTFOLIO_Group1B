\section{Summary}
The main concept that Zimmer faces out with this paper is the impact of \textit{Vehicle Safety Communication} (VSC) on the norms that can be used to "decompose" the concept of privacy. Those are norms of:
\begin{enumerate}
\item \textbf{appropriateness}\label{appr}: what information are appropriate to be divulged in a specific context?
\item \textbf{distribution}\label{disrt}: to what extent should personal information be available and open for consultation? 
\end{enumerate}

For Zimmer, VSC will have a strong impact on those norms: the sharing of personal information, such as GPS coordinates (\ref{appr}), combined with the use of IT, will increase the number of users that will have potentially access to all those information causing a possible shift in public acceptance of a reduction of privacy or to conflicts (\ref{disrt}).

\emph{(116 words)}

\section{Interesting concept}
It was particularly interesting how, in the past, the concept of \textit{"privacy in public"} has been treated and, in some cases, even neglected. The development of IT, in fact, brings with itself a new perspective on privacy in public, as personal data have become way easier to collect. Acceptability of outcomes of the development of technologies such as VSC could be more and more problematic since they disrupt contextual integrity acting both on norms of appropriateness and distribution.

Real life situations in which this concept applies are countless: one example can be related to economic transactions; with the advent and the large diffusion of electronic payments, it is more effective to control and avoid unlawful exchanges of money, as every transaction can be easily checked. However, at the same time many concerns rise when motions of abolishing cash and making e-payment mandatory are proposed, since this can lead to the creation of databases that can potentially be used to track users' preferences.

Cookies used on websites are another example: data of different nature can be collected by means of these tools and can be exploited to profile users. Even if these data are mainly used for harmless activities, for instance focused marketing campaigns, apprehension about what kind and how much of personal information can be gathered while browsing the internet is understandable.

\emph{(223 words)}

\section{Trouble understanding}
\subsection{Impact of VSC on contextual integrity}
In the text, Zimmer says that VSC technologies may alter the flow of personal information and threaten the value of public privacy, but he does not explain how. Without considering any security threat in the installed VSC technologies, only a few institution or companies can access those information (at least here in Europe, we do not know if privacy regulations are so strict also in other states, like USA).

\subsection{Privacy level}
What is lacking in the article is the concept that the acceptability of some privacy level and so conceptual integrity may be different in different societies  and may change with time; so what is not acceptable in European countries may not be so in other countries like North Korea and what is not acceptable now not necessarily won't be so forever since values and necessities evolve over time.

\emph{(139 words)}

\emph{(478 total words)}

\newpage
\begin{thebibliography}{99}

\bibitem [Zimmer, 2005]{p1}

Zimmer, M. \textit{Surveillance, Privacy and the Ethics of Vehicle Safety Communication Technologies}. Ethics Inf Technol 7, 201–210 (2005). \url{https://doi.org/10.1007/s10676-006-0016-0}

\end{thebibliography}

\textit{Document write with \LaTeX. Template founded on Overleaf} (\textbf{Copyright (c) 2020 George Kour}).