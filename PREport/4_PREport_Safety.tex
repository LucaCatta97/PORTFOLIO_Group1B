\section{Summary}
Nowadays, the increasing problem of congestion and traffic leads to a greater consideration of \textit{traffic safety} and its moral issues: on the one hand, mobility is a direct application of the individual right of liberty; on the other, leaving people completely free to travel as they wish leads inevitably to an unbearable amount of deaths and injuries.

In this article, five methods are presented to analyse the trade-offs between personal freedom and limitations in favour of safety:
\textit{\begin{itemize}
        \item criminalisation;
        \item paternalism;
        \item privacy;
        \item justice;
        \item responsibility.
        \end{itemize}}
Each one of these has been selected due to its importance in moral philosophy and its ability to be used to categorise more specific problems.

\emph{(114 words)}

\section{Interesting concept}
A very interesting concept of the article is the idea that responsibility of accidents can be associated not only to drivers, but also to systems designers and policy makers. In fact, the way in which infrastructures, vehicles and traffic are managed and organised can make it more difficult for drivers to break the law and adopt dangerous behaviour.

From this assumption, the forward responsibility concept is derived: designers must forecast, as much as possible, the counter effects of the lack of responsibility of users. It must be noticed that often accidents avoiding measures may cause a strict limitation of individual freedom and this is an important side effect that must be taken into account.

Many examples can be found in the traffic field: for instances speed bumps, that prevent drivers to exceed in speed in order to not damage his/her vehicle, or also the imposition of the driving license to operate a vehicle so to assure the user's knowledge of traffic rules. 
Other measures, like \textit{Intelligent Speed Adaptation} (ISA) or alcohol interlocks could be even more effective. On the one hand, they physically prevent many accidents occasions but, on the other hand, they do this at higher individual liberty expense. In fact, since they are devices directly applied to the vehicle (and the driver is unable to disengage them) , the public may have a strong resistance to their acceptance.

\emph{(229 words)}

\section{Trouble understanding}
\subsection{When can we apply paternalism?}
A concept we had trouble understanding is when paternalistic measures can be applied. The assumption is to decide for people that are not considered able to always take the best decision for themselves or others; for example, allowing people to swim when a lifeguard is not present may harm both the person risking to drown and the people that need to rescue him/her.
However, it is not clear how to identify the threshold, or even if it is possible to identify it, for how much someone can be considered able to take reasonable decisions.

\emph{(94 words)}

\emph{(406 total words)}

\newpage
\begin{thebibliography}{99}

\bibitem[Fahlquist, 2009]{p1}
Fahlquist, J. N. (2009)
\textit{Saving lives in road traffic-ethical aspects.}

\textbf{Journal of Public Health, 17(6)}

\bibitem[Smids, 2018]{p2}
Smids, J. (2018)
\textit{The moral case for intelligent speed adaptation.}

\textbf{Journal of Applied Philosophy, 35(2)}

\end{thebibliography}

\textit{Document write with \LaTeX. Template founded on Overleaf} (\textbf{Copyright (c) 2020 George Kour}).